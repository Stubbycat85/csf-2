\documentclass{article}

\usepackage[osf]{mathpazo} % Use Palatino / Euler fonts
\usepackage{epsfig}
\usepackage{graphicx}% Include figure files

\usepackage{listings}
\usepackage{color}

\usepackage{enumerate}

\definecolor{dkgreen}{rgb}{0,0.6,0}
\definecolor{gray}{rgb}{0.5,0.5,0.5}
\definecolor{mauve}{rgb}{0.58,0,0.82}

\lstset{frame=tb,
  language=Java,
  aboveskip=3mm,
  belowskip=3mm,
  showstringspaces=false,
  columns=flexible,
  basicstyle={\small\ttfamily},
  numbers=none,
  numberstyle=\tiny\color{gray},
  keywordstyle=\color{blue},
  commentstyle=\color{dkgreen},
  stringstyle=\color{mauve},
  breaklines=true,
  breakatwhitespace=true
  tabsize=3
}

\title{Computer Science Foundations\\ Puzzle-Solving Workshop and Seminar\\
\large{Episode 2---October 7}\\
Fall 2013}

\author{Paul Pham and Neal Nelson}

\begin{document}

\maketitle

Welcome to the Puzzle-Solving Workshop and Seminar for
Computer Science Foundations.

In today's workshop (the first hour), we will go over the
solutions to last week's problems (Gauss's problems and
Ada's problems) and grade each other's work. We will
also receive one new problem to test the propositional
logic skills you learned in Discrete Math lecture last week.

In today's seminar (the second hour), we will generate ideas
in groups for your academic statement, then we will freewrite.
To generate ideas, we will divide up into groups of five again
with the three roles we discussed last time: timekeeper,
notetaker, and moderator.

%%%%%%%%%%%%%%%%%%%%%%%%%%%%%%%%%%%%%%%%%%%%%%%%%%%%%%%%%%%%%%%%%%%%%%%%%%%
\section{Problem 1}

Now we will test how general your solution was to Gauss's problem
from Workshop 1. Write the sum of all the \emph{even} numbers
from $2$ to $n$.

\begin{equation}
\Theta =  2 + 4 + 6 + \ldots + n
\end{equation}

%%%%%%%%%%%%%%%%%%%%%%%%%%%%%%%%%%%%%%%%%%%%%%%%%%%%%%%%%%%%%%%%%%%%%%%%%%%
\section{Problem 2}

This builds on Ada's problem from Workshop 1. Consider the final
version of that problem where Ada uses three boolean inputs to
determine whether or not to bike to school:
\texttt{thunderAndLightning}, \texttt{classAt10OClock},
and \texttt{isTired}.

\begin{enumerate}[(a)]
\item
Given that Ada chooses to bike to school, what are all the
possible values of the three input variables that can lead
to this outcome?
\item
What is the equivalent boolean logic expression (using
\texttt{and}, \texttt{or}, \texttt{not}) in Python
for expressing Ada's decision-making rules in a single line?
\end{enumerate}

\end{document}