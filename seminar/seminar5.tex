\documentclass{article}

\usepackage[osf]{mathpazo} % Use Palatino / Euler fonts
\usepackage{epsfig}
\usepackage{graphicx}% Include figure files

\usepackage{listings}
\usepackage{color}

\usepackage{enumerate}
\usepackage{graphicx}
\usepackage{url}

\definecolor{dkgreen}{rgb}{0,0.6,0}
\definecolor{gray}{rgb}{0.5,0.5,0.5}
\definecolor{mauve}{rgb}{0.58,0,0.82}

\lstset{frame=tb,
  language=Java,
  aboveskip=3mm,
  belowskip=3mm,
  showstringspaces=false,
  columns=flexible,
  basicstyle={\small\ttfamily},
  numbers=none,
  numberstyle=\tiny\color{gray},
  keywordstyle=\color{blue},
  commentstyle=\color{dkgreen},
  stringstyle=\color{mauve},
  breaklines=true,
  breakatwhitespace=true
  tabsize=3
}

\title{Computer Science Foundations\\ Puzzle-Solving Workshop and Seminar\\
\large{Episode 5---October 28}\\
Fall 2013}

\author{Paul Pham and Neal Nelson}

\begin{document}

\maketitle

Welcome to the Puzzle-Solving Workshop and Seminar for
Computer Science Foundations. You will notice there are
two parts to this thread: \emph{workshop} and \emph{seminar}.
Workshop is meant to strengthen your problem-solving skills,
to help you in the Discrete Math thread as well as future
math and computer science courses. Seminar is meant to
work on your discussion, writing, and creativity skills.
We will alternate between the two different modes.

Today we will be in workshop mode only. We will go over the
solutions to last week's problems (Gauss's problems and
Ada's problems). We will teach other in a special
group exercise that doubles in size each time and
finally includes the entire class.

%%%%%%%%%%%%%%%%%%%%%%%%%%%%%%%%%%%%%%%%%%%%%%%%%%%%%%%%%%%%%%%%%%%%%%%%%%%
\section{Quantified Love}

``Let us talk about love,'' Ada says suddenly as she and Carl are sitting
on the grass, enjoying one of the few sunny afternoons left to them that
autumn.

``Yes, let's'' asks Carl, who thought this was supposed to be a date.

``Consider,'' Ada says, raising her finger as she does when she's about to
pose a logical problem. ``Let's use the logical statement $P(x,y)$ to mean that
$x$ is $y$'s soulmate.''

``Hmmm, do you believe that people have soulmates?''
Carl tries to steer the conversation away from discrete math, which is how all
their conversations end. Perhaps some philosophy would be romantic\ldots
``Or that people even have souls?''

``This is just logic. I'll let other people worry about belief and souls.''
Ada opens her notebook and begins to write.
``We will consider the statement $P(x,y)$ over the domain of all people in
the world.''

Carl stews quietly in his disappointment.

Regardless, satisfy Ada's curiosity by
translating the following quantified propositional logic statements into
English and decide whether they are true, based on your opinion. Justify it,
using normal, non-logical arguments.

\begin{enumerate}[(a)]
\item
$\forall x \exists y P(x,y)$
\item
$\exists x \forall y P(x,y)$
\item
$P(x,y) \leftrightarrow P(y,x)$
\end{enumerate}

%%%%%%%%%%%%%%%%%%%%%%%%%%%%%%%%%%%%%%%%%%%%%%%%%%%%%%%%%%%%%%%%%%%%%%%%%%%
\section{A Puzzle About Fidelity}

Carl wants to show Ada that it is ridiculous to reduce human
relationships to logical statements. To do so, he makes up a
fable that is so absurd, Ada will have to give up in frustration
and see his point.\footnote{Note that Google and Microsoft have
asked a hetero-normative version of this puzzle in past job
interviews.}

This is what he says:

\begin{quote}
A village consists of 100 married couples of a specific form.
Each couple consists of a Jealous Partner and a Sketchy
Partner.
Every Jealous Partner in the village knows if
a Sketchy Partner other than his or her partner has cheated, but does not
know if her or his own partner has.
The village has a law that does not allow for adultery.
Any Jealous Partner who can prove that her or his partner
is unfaithful must chase her or him out of
the village that very day.
The people of the village would never disobey this law.
One day, the fabulous drag queen of the village visits and
announces that exactly one Sketchy Partner has been unfaithful.
At the end of that day, the law must be enforced,
and the law can only be enforced once per day.
\end{quote}

``So, what happens?'' Carl asks, half-smiling.

Ada taps her finger to her lips in curiosity,
``Hmm, that's very interesting.'' Instead of
thinking the fable is absurd, she immediatedly
beings scribbling in her notebook and launches
into it. ``Here are some things we should
consider.''

\begin{enumerate}[(a)]
\item
Make up a propositional logic symbol with two variables
for the sentence ``Sketchy Partner $x$ and Jealous Partner $y$ are married.''
\item
Translate the sentence ``Every Sketchy Partner $x$ in the village is
married to a Jealous Partner $y$ and not to any other Jealous Partner in the village.''
into quantified propositional logic.
We will consider the domain of the sentence over the 100
couples in the village, in this and every other problem below.
\item
Translate the sentence ``Every Jealous Partner $x$ in the village is
married to a Sketchy Partner $y$ and not to any other Sketchy Partner in the village.''
into quantified propositional logic.
\item
Are the previous two logical statements equivalent?
\item
Translate the sentence ``Sketchy Partner x has cheated on his or her Jealous Partner.''
into propositional logic.
\item
Make up a quantified propositional logic statement, using the above
problems,
for the sentence ``Jealous Partner $y$ has evidence that her or his Sketchy Partner has cheated.''
\item
Make up a quantified propositional logic statement, using the above
problems,
for the sentence ``Jealous Partner $y$ chases her or his Sketchy Partner out of the village.''
\item
Express the rule of the village using the previous two problems.
\item
Make up a quantified propositional logic symbol for
the drag queen's incriminating statement:
``Exactly one Sketchy Partner in the
village has cheated on her or his Jealous Partner.''
\item
Using all the previous problems,
write a quantified propositional logic statement that describes
what happens at the end of the day, in answer to Carl's question.
\end{enumerate}

\section{Sets}

``But now,'' Ada continues, despite Carl's exasperation,
``let's assume that the female impersonator---''

``The what?'' Carl interrupts, confused about what is even going on anymore.

``The drag queen. Drags queens are also called female impersonators.
Some of them can be very convincing.''

``You don't mean that\ldots'' Carl trails off in disbelief, his eyes
widening.

``I don't know \emph{what} you're talking about.'' Ada scoffs.
``Let's assume that the drag queen says that \emph{more than one}
Sketchy Partner has cheated.''

\begin{enumerate}[(a)]
\item
Make up a symbol to represent the set of all cheating Sketchy Partners,
and make up another symbol for its size (the number of cheating partners).
\end{enumerate}

``We can also consider the set of sets of the Cheaters that a
Jealous Partner knows about. I wonder what we can describe
using set notation.'' Ada stares into the sky.

Carl decides he must put his brain to use in order to impress
Ada. After all, he added up all the integers from 1 to
100, surely he could figure something out in this case.
Help Carl with the following problems.

\begin{enumerate}[(b)]
\item
Make up a different symbol with a subscript $x$ to represent the
set of Cheating Partners that a Jealous Partner $x$ knows about.
\item
Express the set in part $(a)$ using the sets in part $(b)$,
using the symbols you defined and the symbols for set operations
(difference, union, or complement)
\end{enumerate}

``I wonder,'' says Carl, ``if we can represent the
\emph{non}-cheating Sketchy Partners using my notation.''

``That's a pretty good question, Carl,'' Ada says,
looking at Carl as if noticing him for the first time.
Carl opens his mouth to say something charming, just
as Ada's eye is caught by Elise De Morgan waving from across
the campus.

``Hey Ada,'' Elise says, as she walks up. ``Hey Carl.
What are you guys working on?''

``Nothing you'd care about, De Morgan,
really, see you later,''
Carl tries to get rid of the newcomer.

``Haha, always joking, Carl. But De Morgan is my dad's name.
Just call me Elise.''

``Hey, Elise. Maybe you can help us,'' Ada says.
``Since your dad is so good at rules of inference and Boolean logic.
Did he teach you anything about it?''

``I might remember a thing or two,'' Elisa says, ``let's take a look.''

\begin{enumerate}[(d)]
\item
Express the set of all \emph{non}-Cheating Partners
using your set notation from all previous problems.
\end{enumerate}

Will Carl or Elise win Ada's affection, or someone else
entirely?
Will Ada ever stop thinking about logic?
Unlock the next episode to find out!
To be continued\ldots

\end{document}