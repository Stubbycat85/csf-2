\documentclass{article}

\usepackage[osf]{mathpazo} % Use Palatino / Euler fonts
\usepackage{epsfig}
\usepackage{graphicx}% Include figure files

\usepackage{listings}
\usepackage{color}

\usepackage{enumerate}
\usepackage{graphicx}
\usepackage{url}

\definecolor{dkgreen}{rgb}{0,0.6,0}
\definecolor{gray}{rgb}{0.5,0.5,0.5}
\definecolor{mauve}{rgb}{0.58,0,0.82}

\lstset{frame=tb,
  language=Java,
  aboveskip=3mm,
  belowskip=3mm,
  showstringspaces=false,
  columns=flexible,
  basicstyle={\small\ttfamily},
  numbers=none,
  numberstyle=\tiny\color{gray},
  keywordstyle=\color{blue},
  commentstyle=\color{dkgreen},
  stringstyle=\color{mauve},
  breaklines=true,
  breakatwhitespace=true
  tabsize=3
}

\title{Computer Science Foundations\\ Puzzle-Solving Workshop and Seminar\\
\large{Episode 4---October 21}\\
Fall 2013}

\author{Paul Pham and Neal Nelson}

\begin{document}

\maketitle

Welcome to the Puzzle-Solving Workshop and Seminar for
Computer Science Foundations. You will notice there are
two parts to this thread: \emph{workshop} and \emph{seminar}.
Workshop is meant to strengthen your problem-solving skills,
to help you in the Discrete Math thread as well as future
math and computer science courses. Seminar is meant to
work on your discussion, writing, and creativity skills.
We will alternate between the two different modes.

Today we will be in workshop mode only. We will go over the
solutions to last week's problems (Gauss's problems and
Ada's problems). We will teach other in a special
group exercise that doubles in size each time and
finally includes the entire class.

The following workshop problems are taken from
Sherri Shulman's Discrete Math Workshop on October 8, 2012.


%%%%%%%%%%%%%%%%%%%%%%%%%%%%%%%%%%%%%%%%%%%%%%%%%%%%%%%%%%%%%%%%%%%%%%%%%%%
\section{Seminar Discussion}

Last time, I asked you to read
the following chapter from Lauren Ipsum for today's workshop and seminar:
``A Tinker's Trade'' at \url{http://www.laurenipsum.org/tinker}.

\end{document}