\documentclass{article}

\usepackage[osf]{mathpazo} % Use Palatino / Euler fonts
\usepackage{epsfig}
\usepackage{graphicx}% Include figure files

\usepackage{listings}
\usepackage{color}

\definecolor{dkgreen}{rgb}{0,0.6,0}
\definecolor{gray}{rgb}{0.5,0.5,0.5}
\definecolor{mauve}{rgb}{0.58,0,0.82}

\lstset{frame=tb,
  language=Java,
  aboveskip=3mm,
  belowskip=3mm,
  showstringspaces=false,
  columns=flexible,
  basicstyle={\small\ttfamily},
  numbers=none,
  numberstyle=\tiny\color{gray},
  keywordstyle=\color{blue},
  commentstyle=\color{dkgreen},
  stringstyle=\color{mauve},
  breaklines=true,
  breakatwhitespace=true
  tabsize=3
}

\title{Computer Science Foundations\\ Puzzle-Solving Workshop and Seminar\\
Fall 2013}

\author{Paul Pham}

\begin{document}

\maketitle

Welcome to the Puzzle-Solving Workshop and Seminar for
Computer Science Foundations. The purpose of this workshop
is to help strengthen your mental muscles. You may feel weak at
first, but don't worry. We'll start gently and gradually increase
the weight of the problem, and you'll have super intellectual
strength in no time.

What you use your newfound strength for is up to you, but we
\emph{can} tell you that it will make you a better programmer. Even if
all you want to do is write video games or make a website to track
baby polar bears, you'll have to solve mathematical problems
like whether those asteroids are in range of your lasers or the
distance between two GPS coordinates (latitute and longitude).
This workshop will definitely help you in the Discrete Math thread.

Conversely,
programming will make you better at math! (It's like a virtuous
cycle). After all,
one of the things computers do best is crunch numbers, and with
a program, you can calculate mathematical results faster than
working them out by hand.

So let's get started.

%\section{Learning Styles}

%People learn in many different ways, which is a good thing,
%since there are many different ways to solve the same problem.
%The more people who can work on a problem, the more likely that as
%a group, they can find a solution.

%Today, we'll only discuss two ways of solving a particular math
%problem: a way using symbols and a way using pictures.

%\section{General versus Specific Problems}

%A specific problem uses actual numbers. For example, if you wanted to
%make 3 catfish tacos for you and your friends, you would take all the
%ingredients for one taco (a corn taco shell, some shredded catfish,
%refried beans, salsa, etc.) and multiply them by 3.

%A general problem uses symbols, so that you can plug in \emph{any}
%number of catfish tacos, or whatever, and still solve your problem.
%Problems have an input, like the number $3$ in the specific
%case above, and we can replace it with the variable $n$.

%\section{Problem-Solving Strategies}

%One of the most basic, and my personal favorite, way of solving
%problems is to start with the absolute simplest case. Usually
%$n=2$.

%%%%%%%%%%%%%%%%%%%%%%%%%%%%%%%%%%%%%%%%%%%%%%%%%%%%%%%%%%%%%%%%%%%%%%%%%%%
\section{Problem 1}

One day young Carl Friedrich Gauss was bored in school.
His teacher, wanting to keep his students busy doing a repetitive
task, asked everyone to add up all the natural numbers from 1 to 100.
Although they complained, all the students dutifully set about
adding 1 to 2 to get 3, 3 plus 3 equals 6, 6 plus 4 equals 10, well,
you get the idea. (This was before they had computers).

Everyone did this except Carl Friedrich. At first, he stared
at the problem in confusion, like anyone else.
This is the first step to solving any problem, and it is
important not to skip it. The second step is to use a fancy
symbol (he chooses the Greek letter $\xi$) to represent the
answer that you want.

\begin{equation}
\xi = 1 + 2 + 3 + \ldots + 100
\end{equation}

The $\ldots$ in the equation above represents where you would fall
asleep if you actually tried to list the numbers from 1 to 100.

Can you help Carl Friedrich solve his problem without adding
one hundred numbers together, thereby teaching his teacher a lesson?

Trying using the strategy of enumerating simple examples above,
or try drawing a picture of $\xi$ number of blocks in a suggestive
way.\footnote{Not \emph{that} kind of suggestive. In a way that
\emph{suggests} the right answer, silly.}

If you can solve this problem, you've solved the specific case
of Gauss's problem for $n = 100$. What is the solution for
general $n$?

%%%%%%%%%%%%%%%%%%%%%%%%%%%%%%%%%%%%%%%%%%%%%%%%%%%%%%%%%%%%%%%%%%%%%%%%%%%
\subsection{Optional Sidequest}

If that problem above was too easy for you, Gauss's grouchy
schoolmaster has another one that might cause you to sweat a little more.
(Teachers have a great way of making up new problems on the spot).
But don't worry, this is optional, you don't have to do this to get
credit for the workshop.

What is the sum of the first $n$ squares? A square isn't just someone
who isn't cool, a square is also an integer multiplied by itself.
Let's use a different Greek symbol now, like $\zeta$:

\begin{equation}
\zeta = 1^2 + 2^2 + 3^2 + \ldots + n^2
\end{equation}

You will need your powers of suggestive picture drawing, and the
original solution to Gauss's problem above.

%And if that weren't enough, the teacher gives Gauss a final problem that even
%the teacher doesn't know the answer to.

%What is the sum of the first $n$ $k$th powers? That is, a $k$th power
%is an integer multipled by itself $k$ times. Let's use the symbol
%$\omega$

%\begin{equation}
%\omega = 1^k + 2^k + 3^k + \ldots + n^k
%\end{equation}

\pagebreak

%%%%%%%%%%%%%%%%%%%%%%%%%%%%%%%%%%%%%%%%%%%%%%%%%%%%%%%%%%%%%%%%%%%%%%%%%%%
\section{Problem 2}

Every morning, Ada must decide whether to bike to school or not
(and take the bus instead).
Because she is a very logical woman,
she uses the following rules to help her decide.

\begin{enumerate}
\item
By default, she wants to bike to get exercise and slow global warming.
\item
If there is thunder and lightning, she will not bike.
\item
Unless she has a class at 10:00am, then she will still bike through
lightning and thunder.
\end{enumerate}

Because she has heard that programming can make her life easier,
she writes the following Python program to help her decide. All
the variables in this program are \emph{Boolean}, meaning they can
only take on the values \texttt{True} and \texttt{False}.

\begin{lstlisting}
bikeToSchool = True

if (thunderAndLightning and not classAt10OClock):
	bikeToSchool = False
	
print str(bikeToSchool)
\end{lstlisting}

What is the value of \texttt{bikeToSchool} at the end of the
program for all
possible combinations of values \texttt{thunderAndLightning} and
\texttt{classAt10OClock}? Draw a truth table.

Let's add another rule: if Ada is tired, she will not bike no
matter what. We'll represent this with another Boolean variable,
\texttt{isTired}. How would you modify the program above to
make a decision for Ada's three rules? How many different combinations
of values are there for all three Boolean variables?

\end{document}