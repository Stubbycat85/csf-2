\documentclass{article}

\usepackage[osf]{mathpazo} % Use Palatino / Euler fonts
\usepackage{epsfig}
\usepackage{graphicx}% Include figure files

\usepackage{listings}
\usepackage{color}

\usepackage{enumerate}
\usepackage{graphicx}
\usepackage{url}

\definecolor{dkgreen}{rgb}{0,0.6,0}
\definecolor{gray}{rgb}{0.5,0.5,0.5}
\definecolor{mauve}{rgb}{0.58,0,0.82}

\lstset{frame=tb,
  language=Java,
  aboveskip=3mm,
  belowskip=3mm,
  showstringspaces=false,
  columns=flexible,
  basicstyle={\small\ttfamily},
  numbers=none,
  numberstyle=\tiny\color{gray},
  keywordstyle=\color{blue},
  commentstyle=\color{dkgreen},
  stringstyle=\color{mauve},
  breaklines=true,
  breakatwhitespace=true
  tabsize=3
}

\title{Computer Science Foundations\\ Puzzle-Solving Workshop and Seminar\\
\large{Episode 3---October 14}\\
Fall 2013}

\author{Paul Pham and Neal Nelson}

\begin{document}

\maketitle

Welcome to the Puzzle-Solving Workshop and Seminar for
Computer Science Foundations. You will notice there are
two parts to this thread: \emph{workshop} and \emph{seminar}.
Workshop is meant to strengthen your problem-solving skills,
to help you in the Discrete Math thread as well as future
math and computer science courses. Seminar is meant to
work on your discussion, writing, and creativity skills.
We will alternate between the two different modes.

Today we will be in workshop mode only. We will go over the
solutions to last week's problems (Gauss's problems and
Ada's problems). We will teach other in a special
group exercise that doubles in size each time and
finally includes the entire class.

The following workshop problems are taken from
Sherri Shulman's Discrete Math Workshop on October 8, 2012.

%%%%%%%%%%%%%%%%%%%%%%%%%%%%%%%%%%%%%%%%%%%%%%%%%%%%%%%%%%%%%%%%%%%%%%%%%%%
\section{Problem 1: Logic Symbols}

Ada and Carl Friedrich (Gauss) decide to play a game by only speaking in
propositional logic variables. They define the following
variables to stand for English sentences about the weather,
and then they will use them to reason about facts about
the weather.

\begin{tabular}{|r|c|l|}
$c$ & = & ``It is cloudy.'' \\
$r$ & = & ``It is raining.'' \\
$s$ & = & ``It is sunny.'' \\
$w$ & = & ``It is night.''
\end{tabular}

Ada comes up with three rules, or things that she
believes to be true about the weather.

\begin{itemize}
\item If it is cloudy, it cannot be sunny.
\item If it is not sunny, then it is either cloudy or it is nighttime.
\item If it is rainy, the ground is wet and it is cloudy.
\end{itemize}

\begin{enumerate}[(a)]
\item
Translate each of Ada's rules above into propositional logic,
using the variables $c$, $s$, $n$, and $r$ combined with the
logical operators $\land$, $\lor$, and $\lnot$.
\end{enumerate}

In response, Carl also comes up with three rules, or things that she
believes to be true about the weather.

\begin{itemize}
\item It is raining when the ground is wet.
\item If it's sunny, then its not raining.
\item It only rains at night.
\end{itemize}

\begin{enumerate}[(b)]
\item
Likewise, translate all of Carl's rules into propositional logic.
\end{enumerate}

%%%%%%%%%%%%%%%%%%%%%%%%%%%%%%%%%%%%%%%%%%%%%%%%%%%%%%%%%%%%%%%%%%%%%%%%%%%
\section{Problem 2: Tautologies, Contradictions, and Everything in Between}

Ada and Carl notice a curious thing about logic. Some propositional
logic expressions are true no matter what. If you were to draw out their
truth table, they would always evaluate to \textsf{True}.
These are called \emph{tautologies}. It doesn't matter what the variables
stand for. They could be sentences about weather like in the previous
example, or anything else, or completely meaningless symbols.

Ada and Carl are so excited about tautologies that they decide
to form a club about them.

\includegraphics[width=5in]{xkcd_honor_societies.png}

Moreover, there are logical expressions that are always false
no matter what. These are called \emph{contradictions}.
Then there are expressions which are true or false
depending on their input variables. Most logical
expressions fall in this category.

\begin{enumerate}[(a)]
\item
Create a truth table for each of the following expressions.
Determine whether they are always true (a tautology),
always false (a contradiction), or neither.
\begin{itemize}
\item $(p \lor q) \land (\lnot p \lor q)$
%\item $(p \lor q) \rightarrow (\lnot p \land q)$
\item $p \rightarrow (\lnot p \rightarrow q)$
\item $(\lnot (p \land q)) \leftrightarrow (\lnot p \lor \not q)$
\end{itemize}
\end{enumerate}

In their tautology club, Ada and Carl spend a lot of time separating
all the logical expressions in the world into three categories:
tautologies, contradictions, and everything else. To do this,
they must decide
whether two expressions are logically equivalent, and they need your help.

\begin{enumerate}[(b)]
\item
Ada and Carl need to know whether $p \rightarrow q$ and $\lnot q \rightarrow \lnot p$
are logically equivalent. Ada likes to use truth tables, because they seem
sure and foolproof, while Carl likes to take shortcut and use
logical symbols and operators.

Are these two expressions equivalent?

\begin{itemize}
\item First use Ada's method and write out the truth tables for both expressions.
\item Then use Carl's method and try to transform one expression into the other using
logical equivalence rules that you learned in class.
\end{itemize}

\item
Carl tries to convince Ada that truth tables take much too long for
even small numbers of boolean variables. Without using truth tables,
show that $\lnot p \rightarrow (q \rightarrow r)$ and
$q \rightarrow (p \lor r)$ are logically equivalent. How many rows
would the corresponding truth table have for each expression?

\item
Ada believes that truth tables are the best way to represent
logical expressions because they are complete and visual. Everything
is laid out in neat rows for you to see, and they are always the
same size. Logical expressions are sometimes more compact,
but some expressions are just as long as their truth tables!
Furthermore, it's hard to know whether you've found the shortest
or the \emph{best} logical expression, since it can take on
multiple equivalent forms.
She challenges Carl to find a logical expression corresponding
to the following truth table, and he needs your help.

\begin{center}
\begin{tabular}{|ccc|c|}
\hline
p & q & r & ?? \\
\hline
T & T & T & T \\
T & T & F & F \\
T & F & T & F \\
T & F & F & T \\
F & T & T & F \\
F & T & F & T \\
F & F & T & F \\
F & F & F & F \\
\hline
\end{tabular}
\end{center}

\item
\textbf{Optional}
Help convince Ada that logical expressions can often be useful and
more compact than their corresponding truth tables. Generalize your
method in the previous problem to apply to any number of variables,
for any truth table. This is what we mean when we say that
$\lnot$, $\land$, and $\lor$ are a complete (universal) set
of logical connectives.
\end{enumerate}

%%%%%%%%%%%%%%%%%%%%%%%%%%%%%%%%%%%%%%%%%%%%%%%%%%%%%%%%%%%%%%%%%%%%%%%%%%%
\section{Reading for Discussion Next Time}

Read the following chapter from Lauren Ipsum for next Monday's workshop/seminar:
``A Tinker's Trade'' at \url{http://www.laurenipsum.org/tinker}.
Come prepared to discuss algorithms, loops, notions of
``sameness'' and ``differentness'' between two pieces of code,
and how to combine two different pieces of code into the same one.

\end{document}