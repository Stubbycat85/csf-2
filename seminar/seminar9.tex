\documentclass{article}

\usepackage[osf]{mathpazo} % Use Palatino / Euler fonts
\usepackage{epsfig}
\usepackage{graphicx}% Include figure files

\usepackage{listings}
\usepackage{color}

\usepackage{enumerate}
\usepackage{graphicx}
\usepackage{url}

\definecolor{dkgreen}{rgb}{0,0.6,0}
\definecolor{gray}{rgb}{0.5,0.5,0.5}
\definecolor{mauve}{rgb}{0.58,0,0.82}

\lstset{frame=tb,
  language=Java,
  aboveskip=3mm,
  belowskip=3mm,
  showstringspaces=false,
  columns=flexible,
  basicstyle={\small\ttfamily},
  numbers=none,
  numberstyle=\tiny\color{gray},
  keywordstyle=\color{blue},
  commentstyle=\color{dkgreen},
  stringstyle=\color{mauve},
  breaklines=true,
  breakatwhitespace=true
  tabsize=3
}

\title{Computer Science Foundations\\ Puzzle-Solving Workshop and Seminar\\
\large{Episode 9---December 2}\\
Fall 2013}

\author{Paul Pham and Neal Nelson}

\begin{document}

\maketitle

Welcome to the Puzzle-Solving Workshop and Seminar for
Computer Science Foundations. You will notice there are
two parts to this thread: \emph{workshop} and \emph{seminar}.
Workshop is meant to strengthen your problem-solving skills,
to help you in the Discrete Math thread as well as future
math and computer science courses. Seminar is meant to
work on your discussion, writing, and creativity skills.
We will alternate between the two different modes.

%%%%%%%%%%%%%%%%%%%%%%%%%%%%%%%%%%%%%%%%%%%%%%%%%%%%%%%%%%%%%%%%%%%%%%%%%%%
\section{Friends When You Need Them}

Elise has invited Carl and Ada over to her house to work on their
homework. Carl, wanting to spend time alone with Ada, but not sure
how to get rid of Elise, reluctantly agreed.
They are sitting in the garage, where each of them has
set up their own ideal work environment. Carl is standing by a
chalkboard covered in equations, working out his discrete math
homework in a
mathematically elegant and rigorous way. Ada sits at a bench,
hammering and forging and soldering the parts to her pet
computer, which she calls Sparky. When it's finished, Sparky
will run Python programs and help her finish her programming
homework.

Elise slumps in a beanbag
chair on the ground, sketching doodles in her notebook with a
thick black pen. The doodles are a storyboard for her new zine,
which she will take to the comics store to sell when she's
done. If we were to peek over her shoulder, we would
discover that Ada, Carl, and Elise are themselves characters
in her graphic novel, having adventures, mostly related to
discrete math, computer programming, and puzzle-solving.

``Hey honey, I brought you and your friends sandwiches,''
her dad says, poking his head through the garage door.
``I thought you might be hungry.''

``Thanks, Dad, you're the best,'' Elise says, getting up
to hug her dad.

``Yeah, thanks Mr. de Morgan,'' both Carl and Ada say as
they dig into the sandwiches. ``Math is hard.''

``Well, I'm just glad Elise finally has some friends.
It's been lonely for her since her mother died. For both
of us.'' Mr. de Morgan pats his daughter on the head and
turns to her friends. ``I rest a little easier knowing
she can rely on you two.''

Carl and Ada exchange a confused look before turning
back. ``Er, yeah, we'll totally help Elise pass this
discrete math class.''

``Dad, stop being weird, you're embarrassing me,''
Elise pretends to be angry before shooing her dad out.
``Sorry guys, he always makes a big deal when I bring
friends home.''

%%%%%%%%%%%%%%%%%%%%%%%%%%%%%%%%%%%%%%%%%%%%%%%%%%%%%%%%%%%%%%%%%%%%%%%%%%%%
\section{Perfect Numbers}

``I'm stuck---'' Carl finally says, turning to Ada. Always
surprised by her beauty, he stammers for a bit, trying
not to stare like a creep. 
Ada, however, is also
vainly admiring her reflection in one of Sparky's polished
sideplates, smiling at herself.
``Um, er, do you know anything about perfect numbers?''

``Hmm? No, but I do like things that are perfect.''
Ada turns to face him, twirling her hair around a finger.
``Tell me about them.''

\begin{quote}
A perfect number is a positive integer which is equal
to the sum of all of its factors (divisors), including $1$ but
not itself. As an example, the first perfect number
is $6 = 1 + 2 + 3$.
\end{quote}

``But I can't find a formula for finding arbitrarily
many perfect numbers,'' Carl says.

``You don't need a formula. You just need to
write a program.'' Ada finishes screwing in
her computer's sideplates. ``Sparky could do it
for you. I just have to write a Python program
first.''

She flips a switch, and Sparky whirs to life. 
He clacks and spins and grrs and barks with
pseudo-lifelike precision. ``Woof,'' he blurts
out. Ada pats him on the head. ``Good boy.''

\begin{enumerate}
\item[\textbf{Problem 1}] Write a Python program to check the
numbers from $1$ to $100,000$ to determine if any
of them are perfect numbers. Print out the numbers
that \emph{are} perfect. If you don't know
Python,
use the pseudo-code
described in Section 3.1 of your discrete math
textbook by Rosen. Then partner up with someone who does
know Python to run the program and check your work.
\end{enumerate}

Run your program on Sparky (er, or on a website such
as PythonAnywhere or repl.it).

%%%%%%%%%%%%%%%%%%%%%%%%%%%%%%%%%%%%%%%%%%%%%%%%%%%%%%%%%%%%%%%%%%%%%%%%%%%%
\section{A Mystery of Sorts, in the Library}

All of a sudden, a loud crash interrupts them from upstairs.
Exchanging an alarmed glance, the four of them drop their
chalk / screwdriver / notebook / bolts and dash upstairs to discover
the source of the noise.

``Dad?'' Elise calls out loudly, almost in a panic, as they
search the house for her father. ``Dad?!''

``Bark,'' says Sparky.

``Of course, Sparky! Here, in the library,'' Ada says calmly, noticing the open
door and the sound of fluttering pages through it.
Upon entering the
room full of books, they find a broken window, the curtains
billowing in the wind. Several tables and chairs are
knocked over, with books scattered on the ground, as if a
great struggle occurred. A leather armchair sits next to the
still roaring fireplace, where Elise's father was fond of sitting.
On the endtable, an open book has pages fluttering in
the wind. While Carl and Ada scout the room looking for
clues, Elise picks up her father's pipe, which is marking
a place in the book.

``Oh no\ldots'' Elise breathes in fear, as Carl and Ada
gather around. There on the page, in rushed handwriting,
is a note:

\begin{quote}
My dearest Elise,

As you know, I have long suspected that the computers and
robots in
our world are becoming more networked, more controlled by
a sinister power which I have come to call the One Machine.
Its agents may be everywhere, looking for me now.
If you cannot find me, I have left to protect you.
Do not try to find me, it is too---
\end{quote}

Here the note ends in a scrawled line that drags to the
edge of the page, and they find a fountain pen on the
ground.

``He always thought this might happen. He even made up a
secret code that only the two of us knew about. I'd never
thought we'd have to use it.'' Elise finishes reading
the note, her hands falling down in despair.
``We \emph{have} to go after him.''

``But\ldots but your father explicitly said not to,''
Carl reminds her nervously. ``He was going to say the
word \emph{dangerous} next, I'm sure of it.'' He
gestured to the broken furniture. ``Look, there was
a fight between your father and the agents of the One
Machine.''

``No, that was part of our code. When he says not to
come after him, he means the exact opposite!''
Elise is about to cry. ``But I can't do this alone, guys.
I'm scared.'' For some reason, both she and Carl
look to Ada next to see what she will do.

``I agree with Carl,'' Ada says at last, which gave him a
gratified look. ``Based on this apparent struggle, it
looks like Mr. de Morgan was kidnapped or forced to
flee. This will indeed be dangerous.'' Carl nods
vigorously.
She examines the note. ``I also agree with
Elise,'' which causes Carl to frown a little.
``We should go after him.''

``Oh thank you!'' Elise hugs Ada suddenly. Then they
both pause awkwardly and withdraw, but with slight
smiles.

``Ahem!'' Carl interrupts them. ``All right, I'll go
with you. I'm sure you'd do the same for me, although
\emph{my} father is much too smart to get mixed up
with sinister, all-powerful sentient computers. So
how do we find him?''

``Each of the books in my father's library has a
volume number. He said he would write one letter inside
the front cover of each book, when they're all in
order, telling me where to find him. But they're all
scattered around the floor! They'll take forever to sort.''

``Not necessarily. We've been learning sorting algorithms
in discrete math class. I'll program Sparky to simulate
which one would be the fastest for us.''

Sparky wags his robotic tail.

Listed below are the volume numbers of the books lying
scattered on the library floor, along with the single letter
written on the inside of each book's cover. The volume number is
shown before the colon and the letter is shown after the colon,
in each pair, or \emph{item}. There are 35 items
altogether. Note that this is not exactly a Python dictionary, because
order matters here (remember, this is the physical order of the books
as they would look on the floor to our characters.)

\begin{quote}
\texttt{
6:t 30:l 5:f 27:e 12:s 34:s 24:t 32:n 25:i 10:e 8:e 20:n 17:r
33:d 21:d 3:p 1:h 28:i 15:w 22:u 2:i 26:v 7:h 19:i 14:o 16:a
9:s 23:c 29:s 0:s 4:o 18:d 11:u 31:a 13:t
}
\end{quote}

Write a Python program to sort this list using bubble sort and insertion
sort given in pseudo-code in your Rosen textbook (Chapter 3), and then run it
on Sparky (again, PythonAnywhere or repl.it. At least, until you can build
your own robotic dog that understands Python.)

\begin{enumerate}
\item[\textbf{Problem 2}] How many comparisons (between two items) will be necessary to fully
sort the list using bubble sort?
\item[\textbf{Problem 3}] How many comparisons (between two items) will be necessary to fully
sort the list using insertion sort?
\item[\textbf{Problem 4}] What is the scrambled message encoded by these letters?
That is, where should Ada, Elise, and Carl go to find Mr. de Morgan?
\end{enumerate}

\end{document}