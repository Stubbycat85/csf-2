\documentclass{article}

\usepackage[osf]{mathpazo} % Use Palatino / Euler fonts
\usepackage{epsfig}
\usepackage{graphicx}% Include figure files

\usepackage{listings}
\usepackage{color}

\definecolor{dkgreen}{rgb}{0,0.6,0}
\definecolor{gray}{rgb}{0.5,0.5,0.5}
\definecolor{mauve}{rgb}{0.58,0,0.82}

\lstset{frame=tb,
  language=Java,
  aboveskip=3mm,
  belowskip=3mm,
  showstringspaces=false,
  columns=flexible,
  basicstyle={\small\ttfamily},
  numbers=none,
  numberstyle=\tiny\color{gray},
  keywordstyle=\color{blue},
  commentstyle=\color{dkgreen},
  stringstyle=\color{mauve},
  breaklines=true,
  breakatwhitespace=true
  tabsize=3
}

\title{Computer Science Foundations\\ Puzzle-Solving Workshop and Seminar\\
Optional Bonus Material\\
\large{Episode 1---September 30}\\
\large{Fall 2013}}

\author{Paul Pham and Neal Nelson}

\begin{document}

\maketitle

This version of the handout is for people who already have
some discrete math and programming experience and know how to
do the required problems. You still have to do the
required problems and turn them in, but these optional bonus problems
are provided to stretch your thinking even further.

Solutions to them will reflect positively on your evaluation.

However, in this program, we are cultivating a community of
learners and teachers. Learning from and teaching your
peers and fellow students can be way more effective than
listening to instructors. Therefore, this handout will also
contain tips for you to teach one of your classmates how to
do the problem.

\emph{You} will be an assistant teacher
in this situation, and your job will be to help your
fellow student have an \emph{aha} moment of
realization or discovery. Remember the first time you
figured out something difficult, something you didn't
know how to do before and even thought was impossible?
Overcoming that situation
brings an awesome feeling that everyone should have.

The general philosophy to keep in mind is that a teacher
must meet a student where he or she is at. Try figuring
out where your classmate is stuck, and then providing
\emph{just enough} of a hint to push them over their
difficulties. In the weight-lifting analogy, you are
a spotter, or an assistant trainer. The weight
may be too heavy for your classmate to lift on his or
her own, but by providing a slight pull, you can help
them do it. This costs you very little physical / intellectual
effort; at that point, you are mostly being a cheerleader
and providing encouragement.

Bring plenty of paper and something to write with!

\subsection{Instructions}

\begin{enumerate}
\item
Work on these optional bonus problems. If you solve them,
great! Turn them in. If you can't solve them after 15
minutes or so, also great! In any case, stop and move to the next step.
\item
Read the section \textsf{Problem 1 Hints} below. Then look
up and see which of your classmates has their hands raised
for help. Pick one, go over with pencil and pen,
then commence tutoring.
\end{enumerate}

%\section{Learning Styles}

%People learn in many different ways, which is a good thing,
%since there are many different ways to solve the same problem.
%The more people who can work on a problem, the more likely that as
%a group, they can find a solution.

%Today, we'll only discuss two ways of solving a particular math
%problem: a way using symbols and a way using pictures.

%\section{General versus Specific Problems}

%A specific problem uses actual numbers. For example, if you wanted to
%make 3 catfish tacos for you and your friends, you would take all the
%ingredients for one taco (a corn taco shell, some shredded catfish,
%refried beans, salsa, etc.) and multiply them by 3.

%A general problem uses symbols, so that you can plug in \emph{any}
%number of catfish tacos, or whatever, and still solve your problem.
%Problems have an input, like the number $3$ in the specific
%case above, and we can replace it with the variable $n$.

%\section{Problem-Solving Strategies}

%One of the most basic, and my personal favorite, way of solving
%problems is to start with the absolute simplest case. Usually
%$n=2$.

\section{Optional Problem 1a}

If Problem 1 is too easy for you, Gauss's grouchy
schoolmaster has another one that might cause you to sweat a little more.
(Teachers have a great way of making up new problems on the spot).
But don't worry, this is optional, you don't have to do this to get
credit for the workshop.

What is the sum of the first $n$ squares? A square isn't just someone
who isn't cool, a square is also an integer multiplied by itself.
Let's use a different Greek symbol now, like $\zeta$:

\begin{equation}
\zeta = 1^2 + 2^2 + 3^2 + \ldots + n^2
\end{equation}

You will need your powers of suggestive picture drawing, and your
original solution to Gauss's problem.

%%%%%%%%%%%%%%%%%%%%%%%%%%%%%%%%%%%%%%%%%%%%%%%%%%%%%%%%%%%%%%%%%%%%%%%%%%%
\subsection{Optional Problem 1b}

And if that weren't enough, the teacher gives Gauss a final secret
extra challenge problem, which nevertheless is a natural
generalization to the problems above.

What is the sum of the first $n$ $k$th powers? That is, a $k$th power
is an integer multipled by itself $k$ times. Let's use the symbol
$\omega$

\begin{equation}
\omega = 1^k + 2^k + 3^k + \ldots + n^k
\end{equation}

%%%%%%%%%%%%%%%%%%%%%%%%%%%%%%%%%%%%%%%%%%%%%%%%%%%%%%%%%%%%%%%%%%%%%%%%%%%
\section{Problem 1 Hints}

Find out whether the person likes to think with symbols or with
pictures. Follow the appropriate subsection below.

\subsection{Thinking in Symbols}

If they like to think with symbols, write out the numbers
$1,2,3,\ldots$ on the left side of a page, and $\ldots, 98, 99, 100$
on the right side of the page, with a blank in between. Ask
them if they can see a pattern. If they can't, add the numbers $4$
and $97$ to the left and right of the page, respectively.
If they still can't see a pattern, ask them just to add the first
and last number, then add the second and next-to-last number.

Once they understand that each pair has the same sum, pause here and
see if they can take the next step on their own.

If they can't, ask them how many such pairs there are, if you were
to finish writing out all the numbers from $1$ to $100$.

At this point, you should stop giving them hints and let them
ponder the rest of the problem on their own.

\subsection{Thinking in Pictures}

Sitting \emph{opposite} (across a table) from your classmate:
Draw out one square, then two squares underneath that one,
then three squares underneath that one. Label the first row
with a $1$, the second row with a $2$, and so on. You should be
forming a (jagged) right-triangle that lines up on the left.

Ask your classmate if they can think of a simple way to
count all the squares, simply by using the height and
width of the triangle.

Pause here for a few minutes to let them think about it.

If they still don't see the pattern:
ask your classmate to draw the same figure as you, upside down and off to
one side from your drawing, on the same piece of paper.
Then ask them the same question.

At this point, you should stop giving them hints and let them
ponder the rest of the problem on their own.

\end{document}